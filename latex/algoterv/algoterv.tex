\documentclass{homework}
\graphicspath{{./media/}}
\author{Soós Ákos}
\class{Algoritmusok tervezése és elemzése 1.}
\date{\today}
\title{Házi feladat 5.}

\begin{document} 
\maketitle
\setstretch{1.2}

\hfill \break

\begin{problem}
    \begin{proof}[Megoldás]
        A táblázat:
        $$
        \begin{bmatrix}
             0 &  0 &  0 &  0 &  0 &  0 &  0 &  0 &  0 &  0 &  0 \\
             0 &  0 &  0 &  8 &  8 &  8 &  8 &  8 &  8 &  8 &  8 \\
             0 &  3 &  3 &  8 &  8 & 11 & 11 & 11 & 11 & 11 & 11 \\
             0 &  3 &  4 &  8 &  8 & 11 & 12 & 12 & 15 & 15 & 15 \\
             2 &  3 &  4 &  8 & 10 & 11 & 12 & 13 & 15 & 16 & 16 \\
             2 &  3 &  4 &  8 & 10 & 12 & 12 & 14 & 18 & 19 & 21
        \end{bmatrix}
        $$
        Tehát 21 értékű tárgyat tudunk bepakolni a hátizsákba, ezt az 1., 2., 5. elemmel érhetjül el.
    \end{proof}
\end{problem}
\hfill \break

\begin{problem}
    \begin{proof}[Megoldás]
        
    \end{proof}
\end{problem}

\end{document}